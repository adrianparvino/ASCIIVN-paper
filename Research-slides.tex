% Created 2017-10-20 Fri 03:44
% Intended LaTeX compiler: pdflatex
\documentclass[presentation]{beamer}
\usepackage[utf8]{inputenc}
\usepackage[T1]{fontenc}
\usepackage{graphicx}
\usepackage{grffile}
\usepackage{longtable}
\usepackage{wrapfig}
\usepackage{rotating}
\usepackage[normalem]{ulem}
\usepackage{amsmath}
\usepackage{textcomp}
\usepackage{amssymb}
\usepackage{capt-of}
\usepackage{hyperref}
\newif\ifexport
\usepackage[natbibapa]{apacite}
\usetheme{default}
\author{Adrian Parvin D. Ouano}
\date{\today}
\title{Bactrians and camels A verification of the prior work of \cite{dehnadi2006camel} in the context of University of San Carlos -- Talamban Campus Grade 12}
\hypersetup{
 pdfauthor={Adrian Parvin D. Ouano},
 pdftitle={Bactrians and camels A verification of the prior work of \cite{dehnadi2006camel} in the context of University of San Carlos -- Talamban Campus Grade 12},
 pdfkeywords={},
 pdfsubject={},
 pdfcreator={Emacs 25.3.1 (Org mode 9.0.9)}, 
 pdflang={English}}
\begin{document}

\maketitle

\begin{frame}[label={sec:org8527f9c}]{Introduction}
\begin{block}{What is this study about?}
A verification of the methodology of \cite{dehnadi2006camel}, 
specifically on their use of the equality symbol.
\end{block}

\begin{block}{Why was this conceived?}
The conclusion of \cite{dehnadi2006camel} points to the possibility that not everyone can program,
however, their methodology appears to be flawed.
\end{block}
\end{frame}

\begin{frame}[label={sec:orgb13927a}]{Importance of the Study}
The success of this study will contribute to Computer Science Education.

\begin{description}
\item[{H\(_{\text{0}}\)}] Their research remains to not disproven and thus may still
prove to be true, thus an additional methodology may actually
exist.
\item[{H\(_{\text{a}}\)}] Their research is disproven and thus may be proven to be
false, thus perhaps anyone can actually learn programming.
\end{description}
\end{frame}

\begin{frame}[label={sec:orgdf3c111}]{Research Design and Methodology}
\begin{block}{Experimental}
A comparison between equality symbol (=; control) and arrow symbol (<-; treatment).
\end{block}
\begin{block}{Examination type}
\end{block}
\end{frame}
\begin{frame}[label={sec:org99600fa}]{Results and Discussion}
\begin{table}[htbp]
\caption{\label{tab:orgca911f9}
Results}
\centering
\begin{tabular}{lrrrr}
\hline
\hline
 & Consistent & Inconsistent & Blank & Total\\
\hline
Equality symbol (=) & 3 & 17 & 14 & 37\\
Arrow symbol (<-) & 2 & 14 & 14 & 33\\
Total & 5 & 31 & 34 & 70\\
\hline
\hline
\end{tabular}
\end{table}

Fisher's exact test of independence (\(\alpha\) = 0.05)
\begin{description}
\item[{All data}] p = 0.9339
\item[{No blanks}] p = 1
\item[{One-tailed}] p = 0.7532
\end{description}
\end{frame}

\begin{frame}[label={sec:org3b81037}]{Conclusion}
The null hypothesis therefore isn't rejected, and
this paper does not prove that their methodology is flawed.
\end{frame}

\begin{frame}[label={sec:org13392f1}]{References}
\bibliography{Research}
\bibliographystyle{apacite}
\end{frame}
\end{document}
