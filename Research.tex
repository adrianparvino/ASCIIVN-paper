% Created 2017-10-14 Sat 20:57
% Intended LaTeX compiler: pdflatex
\documentclass[12pt]{article}
\usepackage[utf8]{inputenc}
\usepackage[T1]{fontenc}
\usepackage{graphicx}
\usepackage{grffile}
\usepackage{longtable}
\usepackage{wrapfig}
\usepackage{rotating}
\usepackage[normalem]{ulem}
\usepackage{amsmath}
\usepackage{textcomp}
\usepackage{amssymb}
\usepackage{capt-of}
\usepackage{hyperref}
\newcommand\nl{\\}
\newif\ifexport
\usepackage[final]{pdfpages}
\usepackage[margin=1in]{geometry}
\usepackage[natbibapa]{apacite}
\usepackage{usebib}
\usepackage{indentfirst}
\usepackage{fancyhdr}
\usepackage{glossaries}
\usepackage{titlesec}
\usepackage{tocloft}
\usepackage{etoc}
\usepackage{verbatim}
\setglossarysection{subsection}
\makeglossaries
\bibinput{Research}
\exporttrue
\ifexport \usepackage{fontspec} \fi
\ifexport \setmainfont{Arial} \fi
\ifexport \renewcommand{\baselinestretch}{2} \fi
\ifexport \titleformat{\section}{\center \bf}{}{0in}{} \fi
\ifexport \titleformat{\subsection}{\bf}{}{0in}{} \fi
\ifexport \titleformat{\subsubsection}{\bf}{}{0.5in}{} \fi
\ifexport \setlength{\parindent}{0.5in} \fi
\ifexport \renewcommand{\cftdot}{} \fi
\cftsetindents{section}{0em}{2em}
\cftsetindents{subsection}{0.5in}{2em}
\cftsetindents{subsubsection}{1in}{2em}
\makeatletter
\renewcommand{\cftsecpresnum}{\begin{lrbox}{\@tempboxa}}
\renewcommand{\cftsecaftersnum}{\end{lrbox}}
\setlength{\cftsecnumwidth}{0pt}
\renewcommand{\cftsubsecpresnum}{\begin{lrbox}{\@tempboxa}}
\renewcommand{\cftsubsecaftersnum}{\end{lrbox}}
\setlength{\cftsubsecnumwidth}{0pt}
\renewcommand{\cftsubsubsecpresnum}{\begin{lrbox}{\@tempboxa}}
\renewcommand{\cftsubsubsecaftersnum}{\end{lrbox}}
\setlength{\cftsubsubsecnumwidth}{0pt}
\makeatother
\renewcommand\contentsname{\clearpage\begin{center} \normalfont \normalsize \bfseries Contents \end{center}}
\renewcommand\tocloftpagestyle{\thispagestyle{fancy}}
\newcommand{\citetitle}[1]{\usebibentry{#1}{title} \citep{#1}}
\fancypagestyle{plain}{
\fancyhf{}
\renewcommand{\headrulewidth}{0pt}
}
\fancyhf{}
\renewcommand{\headrulewidth}{0pt}
\fancyhead[R]{\thepage}
\newcommand{\sectionbreak}{\clearpage}
\makeatletter
\renewcommand{\maketitle}{
  \includepdf[pages=1]{TitlePage.pdf}
}
\makeatother

\renewcommand{\etocaftertitlehook}{\pagestyle{empty}}
\renewcommand{\etocaftertochook}{\pagestyle {empty}}
\newglossaryentry{equalitysymbol}{name={equality symbol},description={(=)}}
\newglossaryentry{multimodal}{name={multimodal},description={Refers to having more than one hump(the \gls{mode})}}
\newglossaryentry{bimodal}{name={bimodal},description={Refers to having exactly two humps(the \gls{mode}). A special case of \gls{multimodal}}}
\newglossaryentry{normal}{name={normal},description={A curve that can be defined by its standard deviation and mean}}
\newglossaryentry{consistencytest}{name={consistency test},description={A test administered to confirm if the participant has consistent answers on similar topics}}
\newglossaryentry{mathematicalequality}{name={mathematical equality},description={\texttt{x = y}, means that one can substitute all \texttt{x}s in the code for \texttt{y}s and vice versa. Also known as \textit{referential equality}.\\\textit{In the study:} It contrasts with \gls{assignment}}}
\newglossaryentry{assignment}{name={assignment},description={\texttt{x = y} means that \texttt{y} is \gls{evaluated}, and \texttt{x} is set to the result.\\\textit{In the study:} It is used as the paradigm by \cite{dehnadi2006camel}, and contrasts with \gls{mathematicalequality}}}
\newglossaryentry{unimodal}{name={unimodal},description={Having a single hump(\gls{mode})}}
\newglossaryentry{modality}{name={modality},description={The amount of \glspl{mode} in the statistics. (See \gls{multimodal} and \gls{unimodal})}}
\newglossaryentry{mode}{name={mode},description={Local maxima}}
\newglossaryentry{evaluated}{name={evaluated},description={Run a piece of code}}
\newglossaryentry{institution}{name={institution},description={Specifically, educational institutions; e.g. universities}}
\newglossaryentry{hdl}{name={hardware description language},description={A language used for the development and simulation of hardware}}
\newglossaryentry{decouple}{name={decouple},description={Form a new interpretation despite having already having an old contradictory information}}
\author{Anthony Almirante\nl{}Luc Cabellon\nl{}Adrian Parvin Ouano}
\date{\today}
\title{Bactrians and camels A verification and meta-analysis of the supposed double hump in Computer Science in University of San Carlos -- Talamban Campus Grade 12}
\hypersetup{
 pdfauthor={Anthony Almirante\nl{}Luc Cabellon\nl{}Adrian Parvin Ouano},
 pdftitle={Bactrians and camels A verification and meta-analysis of the supposed double hump in Computer Science in University of San Carlos -- Talamban Campus Grade 12},
 pdfkeywords={},
 pdfsubject={},
 pdfcreator={Emacs 25.3.1 (Org mode 9.0.9)}, 
 pdflang={English}}
\begin{document}

\maketitle
\tableofcontents

\clearpage
\pagestyle{fancy}
\section{Introduction}
\label{sec:org312d646}
\subsection{Background of the Study}
\label{sec:org58a4495}
\begin{quote}
The camel has a single hump; The dromedary, two; Or else the other way
around, I'm never sure. Are you?

-- Ogden Nash
\end{quote}

Mistakes happen all the time. 
People were mistaken and believed that the camels' humps functioned as storage of water. 
As members of the society, it is our duty to debunk misconceptions 
that have been spread as gossip and hearsay.
Just as it was a common misconception that the function of the camels' humps,
\cite{dehnadi2006camel} may have also been mistaken in their research
with respect to the existence of double hump with respect to Computer Science\footnote{The researchers would just like to emphasize that Computer Science and Computer Programming are \emph{distinct}; 
however, with exception to special cases, Computer Programming can be considered a prerequisite to Computer Science} 
Education. 
\citep{bornat2014camels,bornat2008mental} claim that 
it is widely believed between computer science professors that 
there are just some people who can't program.
The researchers disagree with that sentiment, as anecdotally, 
many of the Electronic and Communications Engineering students that they had encountered 
at least had the ability to understand and modify code; 
this may be linked to Electronics and Communications Engineering having the tendency to just ``accept'' the rules, 
however, this is purely speculation. 
The ability to accept rules without clear reasoning is
one of the requirements in programming according to \cite{dehnadi2006camel}.
It would be everyone's best interest if it does not exist as 
this implies that with proper education and motivation, 
everyone can indeed program. 
However, if it were true, this would imply that both \glspl{institution} and students 
can both benefit as preliminary screening could be performed 
in order to avoid wasting time and resources on both parties.

\subsection{Statement of the Problem}
\label{sec:org14ef6d5}
Can the students' performance be accurately predicted by the \glspl{consistencytest} 
deviced by the researchers and by \cite{dehnadi2006camel}?

Specifically, the following questions are investigated:
\begin{enumerate}
\item Do the different \glspl{consistencytest} show similar results?
\begin{description}
\item[{H\(_{\text{a}}\)}] The \glspl{consistencytest} show dependence.
\item[{H\(_{\text{0}}\)}] The \glspl{consistencytest} do not show dependence.
\end{description}
\end{enumerate}

\subsection{Literature Review}
\label{sec:org98bdcd8}
The focus of the review lies on \citetitle{dehnadi2006camel} as this
paper is a derivative work.
\subsubsection{\citetitle{dehnadi2006camel}}
\label{sec:org877a0c8}
The researchers found the use of the \gls{equalitysymbol} in the test quite questionable; 
in fact, \cite{dehnadi2006camel} had also acknowledged this and specifically stated 
``[W]e normally excoriate this appalling design decision'' 
and further added that it was, to borrow their words, ``crass idiocy'';
the first statement is once again repeated in \citep{bornat2008mental}. 
The researchers speculate that they continued to use that for the following reasons, 
(a) to test if they can \gls{decouple} their previous learning, and
(b) in order to actually test if they would be able to understand actual programming. 
To add to the confusion, they also continued to use the \gls{equalitysymbol} in the test choices,
which means \gls{mathematicalequality} rather than \gls{assignment}.
Another thing to note is that some programming languages 
(e.g. Haskell\footnote{However, the bound variable still remains to be declared at the left-hand-side.}), 
actually do have \gls{mathematicalequality} as the definition of the \gls{equalitysymbol}. 
As the test isn't available at the time of writing; 
the online test developed in \cite{onlinetest} will serve as the basis of the the paper test which will is attached as \hyperref[sec:orgf107c97]{Appendix A}.

\subsubsection{\citetitle{bornat2008mental}}
\label{sec:org18045dc}
\cite{bornat2008mental} stated in his paper that 
a significant number of novice computer programmers have never learned to program during their University course;
this sentiment is supported by \cite{mccracken2001multi}. 
He then further state that he chose to teach programming by way of formal reasoning \cite[Bornat, 1986 cited in][]{bornat2008mental}
If expert programmers could justify their programs, 
perhaps novices could be taught to program by first learning to formalise their reasoning. 
Examples of which is by reading books on programming languages, analytical thinking and reasoning. 
Reading plenty of resources on programming language and algorithms. 
\cite{murnane1993psychology} related programming to psychological theories, 
specifically Chomsky’s notion of natural language acquisition and Piaget’s theories of education.

\subsubsection{\citetitle{bornat2014camels}}
\label{sec:org97b4cdd}
\cite{bornat2014camels} states in his paper that during the creation of
\citep{dehnadi2006camel}, he was experiencing depression. This resulted
in a confirmation bias that there are students that really could not
learn. However, implications can be found all around the paper that
they believe that their prior claims were not completely invalid; 
a very clear example would be in his own words, 
``I believe that the problem is real but that we don’t understand its causes'' (p. 1)
Furthemore, he cites other researches\cite[Bornat, Dehnadi, \& Simon, 2008;
Dehnadi, 2009; Caspersen, Larsen, \& Bennedsen, 2007, all cited in][]{bornat2014camels}
that use the testing methods they had developed. 
This leads the researchers to be compelled to continue where they left off.

\subsubsection{Synthesis}
\label{sec:org0b2df0d}
The similarity of the studies conducted by \cite{dehnadi2006camel} and
\cite{bornat2008mental} was that both tested the teaching approaches of
the researchers to novices or amateur programmers, and whether if
their methods were adequate enough to teach novice programmers to
catch up to expert programmers in terms of their ability to \gls{decouple}
their learning, and to test if they can actually understand actual
programming as stated in \cite{dehnadi2006camel}. And to teach
programming by way of formal reasoning, and to test that, perhaps
maybe novices could be taught to program by first learning to
formalize their reasoning. And lastly, in \cite{bornat2014camels} it was
stated that the first research which was the \cite{dehnadi2006camel} was
basically made because of the depression the researcher went
through. However, the researchers believe that Bornat isn’t trying to
complete invalidate all their prior claims. Clear examples would be
citing different researches that used the said testing methods,
getting them to the decision to continue what the researchers started.

\subsubsection{Other studies}
\label{sec:org6bf2fc4}
\cite{bricklin2002} made an informal synthesis and concluded with the following: 
(a) immediately observable results are easier to grok; 
(b) less amount of instructions are easier to understand; 
(c) instructions that have a clear effect on the results tend to be more engaging.
Point c might have an implication on this study: 
the \gls{equalitysymbol} in \cite{dehnadi2006camel} was used in 2 contexts;
this might cause confusion within the respondents producing worse results.

\cite{chalk2003improving} created a major change in the curriculum and learning environment.
One of those changes were adding interactive visualization on abstract topics;
this contrasts with claims of \cite{dehnadi2006camel} that different paradigms had been attempted yet 
none resulted in any significant improvement.

\subsection{Importance of the Study}
\label{sec:org04d9945}
Just like our predecessors, \cite{dehnadi2006camel}, 
this research paper aims to quantitatively evaluate the students' understanding of programming. 
An alternative importance was demonstrated where 
one assesses not the students but rather the effectiveness of the curriculum of a given course
such as in the case of \cite{ford2010assessing}.
The success of this study is beneficial whether or not 
the consistency test developed by \citeauthor{dehnadi2006camel} does have predictive capabilities.

If consistency test results do not correlate with programming capabilities,
this hints but not conclusively implies that anyone can be a programmer; specifically, 
this might be beneficial to the industry as programmers are often in demand;
online, there is a general consensus that skilled programmers are deficit.

However, if there a correlation does exist, 
both the \gls{institution} and the students benefit in terms of time and resources.
\Glspl{institution} will be able to focus the workforce on those who have a higher guarantee of learning.

\subsection{Scope and Delimitation}
\label{sec:org3bcc2f4}
The population of the study is constrained only to the Grade 12 Senior High School students of University of San Carlos; 
this is not due to time and resource constraint, but instead, 
the researchers are only interested in the performances of the Grade 12 students since

after graduation from Senior High, they'll have to choose their courses; and
the researchers are specifically targetting University of San Carlos since 
not only is Information Technology and Computer Science taught here,
but also Electronic and Communications Engineering 
which has close ties with Computer Programming
such as in the case of Icarus Verilog \citep{iverilog}, 
which is a \gls{hdl}.

As had been observed by \citep{dehnadi2006camel}, 
some of the participants are expected to
decline either due to lack of free time or even simple disinterest.
The former can become a problem due to a possible bias in the results,
specifically that our population will be skewed to those who are
interested in Computer Science.

Furthermore, although this paper explores the effects of different representations of programming, 
it is not within the scope of the study to identify which programming concepts are difficult to understand. 
As such, the findings of this research should not be used to research to 
dissect teaching curricula into their strengths and weakness; rather, 
it should only be used to quantify its effectiveness.

The instrument might also be susceptible to the same problems as \cite{dehnadi2006camel},
although in this research, an ``arrow'' was used, 
it was in ASCII form and not in Unicode form,
rather than \texttt{<-}, ← should have been used.

Finally, as a derivative of the work of the study by \cite{dehnadi2006camel}, 
the focus lies on two things: 
first, the research aims to assess students with different testing methods and 
next, compared the regression of the different testing methods to one another.


\printglossaries

\section{Methodology}
\label{sec:org0fab3d7}
\subsection{Research Design}
\label{sec:org58fe366}
The design is experimental and correlational in nature.  
The researchers will conduct multiple experiments which include different paradigms in programming. 

\subsection{Participants and Sampling Technique}
\label{sec:orga7a4118}
Fifty samples each are taken for the different examinations. 
There are 9 blocks in University of San Carlos -- Talamban Campus.
Both random and non-random sampling techniques are performed. 
Stratified sampling with the strata as the Grade 12 blocks in University of San Carlos -- Talamban Campus is performed
which is followed by a quota sampling with 5 to 6 students per block.

\subsection{Research Instrument}
\label{sec:orga9a9a1f}
A paper survey is conducted; 
the contents of the paper survey is attached as Appendices \hyperref[sec:orgf107c97]{A} and \hyperref[sec:orgc1ee00a]{B}.

\subsection{Data Gathering Procedure}
\label{sec:orge19ca97}
The researchers will distribute the paper survey to each classroom that is included in the population. 

\subsection{Data Analysis}
\label{sec:orgce9507a}
The \(\chi^{\text{2}}\)-test of independence is performed to verify that 
the model forms a 3x2 contingency table 
with the rows as passing and failing the consistency test,
and the columns as the equality symbol and the arrow symbol.


\section{Results and Discussion}
\label{sec:org474ced4}
Although 101 consistency tests were sent out, only 37 were returned.

\begin{table}[htbp]
\caption{\label{tab:org7c74a6f}
Results}
\centering
\begin{tabular}{lrrrr}
 & Inconsistent & Consistent & Blank & Total\\
\hline
Equality symbol (=) & 2 & 15 & 2 & 19\\
Arrow symbol (<-) & 2 & 12 & 4 & 18\\
Total & 4 & 27 & 6 & 37\\
\end{tabular}
\end{table}

The data shows that the equality symbol (=) and the arrow symbol (<-) have a p-value of 0.6146 (\(\chi^{\text{2}}\) = 0.97368,  df=2),
thus, we fail to reject the null hypothesis and
we cannot claim that the the arrow symbol shows a different distribution than the equality symbol.

\section{Conclusion and Recommendations}
\label{sec:org90c0c9d}

\subsection{Conclusion}
\label{sec:orgc3a0278}
The claims by \cite{dehnadi2006camel} that different paradigms in has not been disproven by this research.
Despite using a more comprehensible symbol, the distribution remains the same.

\subsection{Recommendations}
\label{sec:org5bd515e}
The sample size in this study was greatly reduced due to external factors. 
A more rigorous data collection is thus adviced, amongst other things.
Furthermore, following the footsteps of \cite{dehnadi2006camel}, 
a second test should have been administered in order to 
verify their claims of bimodality.

\raggedright
\bibliography{Research}
\bibliographystyle{apacite}
\addcontentsline{toc}{section}{References}

\section{Appendices}
\label{sec:orgb5d6273}
\subsection{Appendix A}
\label{sec:orgf107c97}
\includepdf[pages=-,pagecommand={},width=\textwidth]{exam.pdf}

\subsection{Appendix B}
\label{sec:orgc1ee00a}
\includepdf[pages=-,pagecommand={},width=\textwidth]{exam2.pdf}
\end{document}
